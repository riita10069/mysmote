\chapter{Conclusion}

I used oversampling and undersampling techniques as an approach to imbalanced data sets.
In addition, I implemented a new oversampling method called MySMOTE.

What I found out strongly is that there is no silver bullet.
Depending on the data set, the appropriate approach varies greatly.
Even for Smote and Tomek Links, which one was superior or not depended on the data set.
There were also cases where Combine worked well even when Smote and Tomek Links did not.
Furthermore, Combine did not produce results that were less successful than Smote or Tomek Links.
We felt that Combine is a very practical algorithm.
MySMOTE, the proposed method in this thesis, was also found to work well on some datasets.
When Smote or Tomek Links could not increase the recall, MySMOTE could greatly increase the recall and improve the AUC in some datasets.
This suggests that MySMOTE may be a useful method for limited applications.

Again, no silver bullet was found in the sampling method.
We found that a more appropriate approach can be approached by considering not only results such as AUC, but also what specifically needs to be improved such as recall and precision, and taking various solutions.

