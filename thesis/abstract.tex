A fundamental problem in supervised machine learning is class imbalance.
For example, consider a training set consisting of 100 cases of two classes.
Consider a training set consisting of 100 cases of two classes, where 5 cases belong to the positive class and 95 cases belong to the negative class.
The 95 cases belong to the negative class. The positive class is the minority class and is substantially smaller than the majority class.
When the classifier tries to learn to distinguish between the two classes, it will have difficulty learning well because there will not be a sufficient number of positive examples.
This is because negative examples dominate. There are various ways to solve this problem.
There are two main methods.
The first is the method of weighting the cases on the positive cases.
The second is to change the data set itself in preprocessing.
In this study, we will consider in depth the method of changing the data set itself in preprocessing.
There are two approaches: oversampling, which increases the number of cases in the minority data, and undersampling, which decreases the number of cases in the majority data.
